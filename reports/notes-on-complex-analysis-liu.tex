% Preamble
\documentclass[11pt,reqno,oneside,a4paper]{article}
\usepackage[a4paper,includeheadfoot,left=25mm,right=25mm,top=00mm,bottom=20mm,headheight=20mm]{geometry}
\input{../texhead-main} % Standard packages, page layout, theorem environments, macros, etc
\input{../texhead-project} % Macros specific to this project.
\usepackage{bbm}
\author{Zhang Liu}
\title{Notes on Topological Data Analysis}
\renewcommand{\runningtitle}{Notes on Topological Data Analysis}
\date{\today}

\begin{document}
\maketitle
\thispagestyle{fancy}

\begin{abstract}
    This document is to serve as a set of notes to fill the gaps in my understanding of Topological Data Analysis relevant to the project. The reference book for this set of notes is \textit{Algebraic Topology} \cite{Hat2002a}, \textit{Topology and Data} \cite{Car2009a}, and \textit{Topology for Computing} \cite{Zom2005a}.
\end{abstract}

% \tableofcontents

\section{Homotopy} \label{sec:homotopy}

\begin{defn}
	Homotopy is a family of maps $f_t: X \to Y$ where $t \in I$ such that $F: X\times I \to Y$ defined by $F(x,t) \mapsto f_t(x)$ is continuous.
\end{defn}

\begin{defn}
	Two maps $f_0, f_1$ are homotopic if $\exists$ a homotopy $f_t$ between $f_0$ and $f_1$.  
\end{defn}

A special case of homotopy is the deformation retraction. 

\begin{defn}
	A deformation retraction of $X$ onto a subspcae $A$ is a homotopy from the identity map of $X$ to a retraction of $X$ onto $A$, $r: X\to X$ such that $r(X) = A$ and $r|_A = \mathbbm{1}$ (or equivalently, retraction is the map $r: X\to X, r^2 =r$). 
\end{defn}

Retraction is the topological analog of projection. To visualize this analogy, we give an example of how some deformation retractions arise from the mapping cylinder.

\begin{defn}
	For a map $f: X\to Y$, the mapping cylinder $M_f$ is the quotient space of the disjoint union $(X \times I) \sqcup Y$.
\end{defn}


\bibliographystyle{amsplain}
{\small\bibliography{../dbrefs}}
\end{document}
