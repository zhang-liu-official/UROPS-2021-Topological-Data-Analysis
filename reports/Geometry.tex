% Preamble
\documentclass[11pt,reqno,oneside,a4paper]{article}
\usepackage[a4paper,includeheadfoot,left=25mm,right=25mm,top=00mm,bottom=20mm,headheight=20mm]{geometry}
\input{../texhead-main} % Standard packages, page layout, theorem environments, macros, etc
\input{../texhead-project}
\usepackage{amssymb,amsmath,amsthm}
\usepackage{xcolor,graphicx}
\usepackage{verbatim}
\usepackage{hyperref}
% Layout of headers & footers
\usepackage{titling}
\usepackage{fancyhdr}
\usepackage{float}
\pagestyle{fancy} \lhead{{\theauthor}} \chead{} \rhead{} \lfoot{} \cfoot{\thepage} \rfoot{}

% Hyphenation
\hyphenation{non-zero}
\title{Research Statement and Background}
\date{Feburary 27, 2021}
\author{Zhang Liu}

\begin{document}
	\maketitle
	\thispagestyle{fancy}
	
	\section{Personal Statement}

	My guiding principle for my career is to find the intersection between what energizes me and what makes an impact. To this end, I have chosen to focus my research interests on topics related to Deep Learning and adjacent fields of Applied Mathemtics as well as Topology. The research problems in these areas fuel up my intellectual energy and empower me because their applications have potential positive social impact.  
	
	 The project topic on the theoretical aspects of deep learning is particularly relevant to my current independent research project on Topological Data Analysis (TDA), which ``has tremendous potential for broadening our exploration and understanding of complex, high-dimensional data spaces" (Medina et al., 2015) \cite{MD2015a}. My project is titled \textit{Persistent Homology of Semantic Spaces}, with the aim to investigate the effectiveness of applying methods in the emerging field of TDA to solve problems that involve natural language data. I am currently studying applied algebraic topology, persistence and simplicial homology. Subsequently, I will investigate several notable applications where topological methods are used to approximate the underlying geometrical patterns in high-dimensional data.  In particular, I will look at the applications involving natural language data and hope to investigate how effective the topological approach is compared to the traditional regression-based tools and why, which is still an open problem (Savle et al., 2019) \cite{Sav2019a}. The skills I am expected to gain from this project include a sound theoretical foundation of TDA and its various applications as well as the expertise in implementing TDA models. The most updated progress can be found at \href{https://gitlab.com/zhang-liu-official/urops-2021-topological-data-analysis}{this GitLab repository}. 
	
	I would also like to find out more about the connections between Machine Learning and Applied Mathematics, which align well with my prior experiences and interests. 
	
	I have completed the widely recognized DeepLearning.ai specialization on Coursera. The course was taught by Professor Andrew Ng from Stanford University and consisted of the following courses: Sequence Models, Neural Networks and Deep Learning, Convolutional Neural Networks, Structuring Machine Learning Projects, Improving Deep Neural Networks: Hyperparameter Tuning, Regularization and Optimization. Having completed this specialization, I was equipped with skills in implementing ML algorithms in Python Jupyter Notebook with Tensorflow and Keras back-end.  
	
	Additionally, I have completed two substantial machine learning projects. From April 2016 to December 2017, I worked on the project Application of Machine Learning in Automatic Sentiment Recognition in Human Speech as part of a research program for a selective group of high school students. This project explored the possibility of applying supervised Machine Learning in recognizing sentiments in English utterances on a sentence level. The project also aimed to examine the effect of combining acoustic and linguistic features on classification accuracy. With this project, I came among the International Finalists in the most prestigious science competition in China, the Dongrun-Yau Science Award organized by Tsinghua University. (Fewer than ten international finalists were selected per year.) The paper for this research project was published on Dec 27, 2018 as part of the Advances in Intelligent Systems and Computing book series (AISC, volume 887). Please refer to \href{https://link.springer.com/chapter/10.1007/978-3-030-03405-4_41}{the paper link} and \href{https://www.youtube.com/watch?v=fwrFHwDvMnU&list=PLS161HmXR6JjDQjE-lja23OKrbkJeO0vW&index=16}{the conference interview} for more details. From Jan 2018 to Jun 2018, I worked full-time as a research intern at Lab of Medical Mechatronics, National University of Singapore. I worked on an independent project on emotion recognition from electroencephalography (EEG) signals, using hybrid deep neural networks as well as transfer learning. Please refer to \href{https://sites.google.com/view/liu-zhang/home}{the project webpage} for more details. 
	
	I also participated in two Applied Mathematics summer projects. From July 2019 to September 2019, I worked on the project Hybrid Analytical-Numerical Integration. This project focused on Julia library for exponential sums and survey of Langer's work. I implemented a Julia library providing for the efficient description and algebra of exponential sums. Exponential sums are exponential polynomials in which each polynomial coefficient is a complex constant. I also surveyed Langer's (1931) paper on the asymptotic locus of zeros of such functions. For research outputs, please refer to my \href{https://www.unifiedtransformlab.com/outputs.html#Zha2019a}{report} and \href{https://www.unifiedtransformlab.com/outputs.html#Zha2019b}{Julia code}. As a continuation of this project, I worked on the project Numerical Exponential Polynomials from May 2020 to August 2020. This project focused on the relative growth of terms in exponential polynomials. In different regions of the complex plane, different terms of an exponential polynomial dominate others. I used analytic and geometric arguments to compare the relative dominance of terms. For research outputs, please refer to my \href{https://github.com/zhangliu6/pdfs/blob/master/rank-terms-liu.pdf}{report} and  \href{https://github.com/zhangliu6/pdfs/blob/master/notes-on-complex-analysis-liu.pdf}{expository notes on Complex Analysis}. 
	
	
	In terms of coursework, I have planned my academic career such that I have a rigorous foundation in  the core areas of both Mathematics and Theoretical Computer Science. 
	
	During my Modern Algebra class,  I have completed two exploration projects where I investigated topics related to abstract algebra and presented on the respective topics in the video format. The first project is ``Classification and Generalization of Plane Isometries" (\href{https://youtu.be/63exo4bGD7k}{the final video presentation} and \href{https://github.com/zhangliu6/course-submissions/blob/main/Modern_Algebra_Exploration_1%20(1).pdf}{slides}). The other is ``Applications of Polynomial Rings in Algebraic Coding Theory" (\href{https://youtu.be/_XOCvLLPfVQ}{the final video presentation}). Taking a step further from what I have learned in this class, I am currently taking an independent reading module on Infinite Group Theory which includes the following topics: free groups and finitely presented groups, group theoretical classes and closure operations, Abelian groups, decomposition of groups, series, calculus of commutators, introduction to soluble groups, introduction to nilpotent groups, introduction to braid groups, and application of braid groups. 
		
	My Functional Programming and Proving class has also been an illuminating experience that showed how mathematical reasoning can be formalized in computer programs. Using the Coq Proof Assistant, we studied an integrated account of specifications, unit tests, implementations, and properties of functional programs, through a variety of examples. This course culminated in a final project where we formalized an interpreter for arithmetic expressions, a compiler from arithmetic expressions to byte code, and an interpreter for byte code (i.e., a virtual machine), and to prove that for any given arithmetic expression, interpreting this arithmetic expression and compiling this arithmetic expression and then running the resulting byte-code program yield the same result, be it a natural number or an error message. (\href{https://github.com/zhangliu6/course-submissions/blob/main/FPP_Midterm%20(4).pdf}{the midterm project report} and \href{https://github.com/zhangliu6/course-submissions/blob/main/midterm-project-liu-new.v}{the midterm project code} as well as \href{https://github.com/zhangliu6/course-submissions/blob/main/FPP_Term_Project_Report.pdf}{the term project report} and \href{https://github.com/zhangliu6/course-submissions/blob/main/term-project-liu-1.v}{the term project code}).
		
	Apart from the formal research and academic involvement, my long-term passion project is to use mathematical and computational methods to implement the Sanskrit linguistic system. For more details, see \href{https://zhangliu.gitbook.io/the-panini-machine/}{the gitbook documentation}. Additionally, I take an avid interest in social computing research and I am currently working as a research assistant  for a project on Regional Cultural Politics of PRC's New Media Technologies in Xinjiang. In this project I analyze and make notes on the databases and websites on the Xinjiang crisis, and academic articles on critical computational/algorithmic studies and digital surveillance. Additionally, the project entails research on Chinese social media websites and apps (Weibo; WeChat; TikTok; Kuaishou, etc) on both official statements and popular sentiments on Xinjiang. I am interested to see how Deep Learning algorithms can be adapted and applied to this interdisciplinary field of social media data analysis and opinion mining in a way that reduces bias.

	 Looking ahead, I would likely anchor my research in the theoretical problems in designing Deep Learning algorithms. To this end, I intend to dedicate my final year to taking relevant graduate courses and working on an independent capstone project in this direction. Hailing from a small liberal arts college in Singapore, I find it challenging to access the resources and the research community specializing in my areas of interest. Therefore, I believe the summer research experience at Cornell will be a valuable opportunity and next step to cement my interests and deepen my insights on the current state of research.  
	
	\section{Academic Courses}
	
	In this section, I will list the relevant academic courses I have taken and am currently taking.
	
	\begin{itemize}
		\item \textbf{YSC1216 Calculus (A)}
		
		\item \textbf{YSC2209 Proof (A)}
	
		\par The course was an introduction to mathematical proofs, including the following topics: board presentation of proof, direct proof, proof by contradiction, proof of contrapositive; inductive proof, quantification, LaTeX \& git; board presentation of proof, inductive proof, branching proof, nonconstructive existence arguments, AM-GM, fundamental theorem of arithmetic; multidimensional inductive proof, binomial theorem, Cauchy-Schwarz inequality; complete inductive proof, mathematical games, elementary combinatorics; infinite sets, diagonal argument; field axioms, construction of complex numbers, infinite sets, constructive existence arguments, descent theorem, construction of number sets; epsilon-delta argument, construction of the real numbers.
		
		\item \textbf{YSC1212 Introduction to Computer Science (A+)}
		
		\par This class was an introduction to the foundations of computer science taught in the functional programming language OCaml (a statically type-checked programming language with an emphasis on expressiveness and safety). The course culminated in the final project on the following topics:
		\begin{itemize}
			\item multiplying integers in a tree using an accumulator
			\item implementing depth-first and breadth-first traversals
			\item indexing strings, arrays, lists, lazy lists, streams, and binary trees
			\item computing the width of a binary tree
			\item implementing a sieve
			\item solving a variant of the knapsack problem
			\item implementing an interpreter, a compiler, and a virtual machine for arithmetic expressions
		\end{itemize}
		Refer to \href{https://github.com/zhangliu6/course-submissions/blob/main/Intro_to_CS_Midterm%20(7).pdf}{the midterm project report} and \href{https://github.com/zhangliu6/course-submissions/blob/main/midterm-codes.zip}{the midterm project code} as well as \href{https://github.com/zhangliu6/course-submissions/blob/main/Term%20Project.pdf}{the term project report} and \href{https://github.com/zhangliu6/course-submissions/blob/main/final-codes.zip}{the term project code}.
		
		\item \textbf{YSC2213 Discrete Mathematics (A+)}
		
		This class introduced the foundations of discrete mathematics, namely combinatorics and graph theory. The class took place in a team problem-solving setting. At the end of the course, I completed the following projects in teams:
		\begin{itemize}
		\item investigating various Catalan objects (\href{https://github.com/zhangliu6/course-submissions/blob/main/Project_1_Pyramid_Scheme_Comments_Zhang_Liu.pdf}{see final report with comments}).
		\item investigating various types of self-avoiding lattice paths of length $n$ in $\mathbb{Z}$ and a specialization of the problem in $a$-by-$b$ lattice (\href{https://github.com/zhangliu6/course-submissions/blob/main/Project_2_Dots.pdf}{see final report with comments}).
		\item optimizing class schedules at our college (\href{https://github.com/zhangliu6/course-submissions/blob/main/Report_3_Ketchup.pdf}{see final report with comments}).
		\end{itemize}

		Textbooks used:	\textit{A Walk Through Combinatorics} by Miklós Bóna, \textit{Proofs that Really Count} by Arthur T. Benjamin and Jennifer J. Quinn, \textit{Pearls in Graph Theory} by Nora Hartsfield and Gerhard Ringel, \textit{Algorithm Design} by Jon Kleinberg and Éva Tardos. 
		
		\item \textbf{YSC2232 Linear Algebra (A-) }
		
		\par The main topics of this course were vector spaces and linear maps between vector spaces. The topics included the basic theory of vector spaces, including bases, dimension, norms, inner products, and direct sums. We studied the theory of linear maps between vector spaces, culminating in the spectral theorems for real and complex operators and the singular value decomposition. We implemented standard matrix calculations and algorithms using the R programming language. The course culminated into the final assignment where I implemented the QR algorithm and Single Value Decomposition of a general matrix. (Refer to \href{https://github.com/zhangliu6/course-submissions/blob/main/Coding1-5.zip}{code}). 
		
		\par Textbooks used: \textit{Linear Algebra Done Right} by S. Axler and \textit{Linear Algebra and Its Applications} by G. Strang.
		
		\item \textbf{YSC3206 Introduction to Real Analysis (A+)}
		
		\par The topics included: Real Numbers, Sequences, Series, Complex Numbers, Exponential  Trigonometric Functions, Metric and Normed Spaces, Limit and Continuity of Functions, Differentiation, Integration.
		
		\item \textbf{YSC3237 Introduction to Modern Algebra (A)}
		
		\par The topics included:	Introduction to Groups, Subgroups \& Cyclic Groups, Cosets \& Homomorphisms, Quotient Groups,		The Isomorphism Theorems,		The Fundamental Theorem of Finite Abelian Groups,		Group Actions \& Enumeration, Conjugation \& The Class Equation,		The Sylow Theorems, Introduction to Rings,		Ideals \& Quotient Rings, Factorization, Primality, \& Irreducibility,		Euclidean, Principal Ideal, and Unique Factorization Domains.
		
		\par Within the class, we also conducted two exploration projects in groups of two. We investigated topics related to abstract algebra and presented on the respective topics in the video format. The projects were:
		
		\begin{enumerate}
			\item Classification and Generalization of Plane Isometries. View the \href{https://youtu.be/63exo4bGD7k}{the final video presentation} and \href{https://github.com/zhangliu6/course-submissions/blob/main/Modern_Algebra_Exploration_1%20(1).pdf}{slides}.
			\item Applications of Polynomial Rings in Algebraic Coding Theory. View \href{https://youtu.be/_XOCvLLPfVQ}{the final video presentation}.
		\end{enumerate}
		
		\par Textbooks used: \textit{Contemporary Abstract Algebra} by Gallian, \textit{Algebra} by Artin, \textit{Abstract Algebra} by Dummit and Foote.  
		
		\item \textbf{YSC3236 Functional Programming and Proving (A+)}
		
		\par Using the Coq Proof Assistant, we studied an integrated account of specifications, unit tests, implementations, and properties of functional programs, through a variety of examples.
		
		\par This course culminated in the final project to formalize an interpreter for arithmetic expressions, a compiler from arithmetic expressions to byte code, and an interpreter for byte code (i.e., a virtual machine), and to prove that for any given arithmetic expression, interpreting this arithmetic expression and compiling this arithmetic expression and then running the resulting byte-code program yield the same result, be it a natural number or an error message.
		
		Refer to \href{https://github.com/zhangliu6/course-submissions/blob/main/FPP_Midterm%20(4).pdf}{the midterm project report} and \href{https://github.com/zhangliu6/course-submissions/blob/main/midterm-project-liu-new.v}{the midterm project code} as well as \href{https://github.com/zhangliu6/course-submissions/blob/main/FPP_Term_Project_Report.pdf}{the term project report} and \href{https://github.com/zhangliu6/course-submissions/blob/main/term-project-liu-1.v}{the term project code}.
		
		\item \textbf{DeepLearning.ai Deep Learning Specialization (Coursera)} 
		
		Refer to \href{https://www.coursera.org/account/accomplishments/specialization/3PAXFXZG7WAZ}{the certificate}.
		
		Course Certificates Completed:
		\begin{itemize}
			\item Sequence Models
			\item Neural Networks and Deep Learning
			\item Convolutional Neural Networks
			\item Structuring Machine Learning Projects
			\item Improving Deep Neural Networks: Hyperparameter Tuning, Regularization and Optimization
		\end{itemize}
		
		\item \textbf{MA4207 Mathematical Logic (IP)}
		
		\par This is an introductory mathematical course in logic. It gives a mathematical treatment of basic ideas and results of logic, such as the definition of truth, the definition of proof and Godel's completeness theorem. The objectives are to present the important concepts and theorems of logic and to explain their significance and their relationship to other mathematical work. Major topics include: Sentential logic, structures and assignments, elementary equivalence, homomorphisms of structures, definability, substitutions, logical axioms, deducibility, deduction and generalization theorems, soundness, completeness and compactness theorems, and prenex formulas.
		
		\par Textbook used: \textit{A Mathematical Introduction to Logic} by Enderton.
		
		\item \textbf{YSC2229 Introductory Data Structures and Algorithms (IP) }
		
		\par This course covers algorithm-and-data-structure basic design techniques and tools, as well as their runtime analysis. Besides, we also study some algorithms that are widely used, such as, divide and conquer, dynamic programming, greedy algorithms, graph algorithms, linear programming, randomized algorithm, etc. We also discuss a few types of data structures, particularly dynamic data structures and their access time analysis: hashing, randomized binary search trees, and an access-time analysis called amortized analysis. The programming languages used for this course are Python and C/C++.
		
		
		\par Textbook used: \textit{Introduction to Algorithms }by Cormen et al.
		
		\item \textbf{MA3288 Advanced Undergraduate Research Opportunities Programme in Science (UROPS) in Mathematics I (IP)}
		
		\par This is a research module on the topic of Topological Data Analysis (TDA). The aim of the project is to investigate the effectiveness of applying methods in the emerging field of TDA to solve problems that involve natural language data. We will study the state-of-the-art method of persistent homology and several notable applications in computational analysis of natural language. In particular, we hope to investigate how effective the topological approach is compared to the traditional regression-based tools and why, which is still an open problem (Savle et al., 2019). In this project, we study the mathematical foundations of the state-of-the-art methods in topological data analysis as well as the existing semantic space models. Further, we aim to investigate the effectiveness of these methods as well as to propose and possibly implement some improvements. 
		
		\par Topics to be studied include: point-set topology, algebraic topology, persistence, homology, simplicial  complex, the persistence algorithms, semantic space models.
		
		\par Some references used so far: \textit{Topology for Computing} by Zomorodian,\textit{ Topology and Data} by Gunnar Carlsson, \textit{Elementary Applied Topology} by Robert Ghrist, \textit{Algebraic topology} by Allen Hatcher.
		
		\item \textbf{YIR3317G Independent Reading and Research (IP)}
		
		\par This is a research module on the topic of infinite group theory. The topics will include: free groups and finitely presented groups, group theoretical classes and closure operations; Abelian groups; decomposition of groups: series; calculus of commutators; introduction to soluble groups; introduction to nilpotent groups; introduction to braid groups - constructed together in class; application of braid groups. This independent reading and research module will culminate into a series of individual presentations as well as proof-based problem solving assignments. 
		
		\par Texbook used: \textit{A Course in the Theory of Groups} by Robinson.
	\end{itemize}

	\section{Research Experiences}
	\begin{itemize}
		
		\item (Jan 2020 -- present) \emph{\textbf{Peer Tutor, \emph{YSC2209 Proof} and \emph{YSC2232 Linear Algebra}.}}
		
		\par As a peer tutor, I assist the instructors and provide academic guidance for students in the core modules in Mathematics, YSC2209 Proof and YSC2232 Linear Algebra. 
		
		\item (Jan 2020 -- present) \emph{\textbf{Persistent Homology of Semantic Spaces.}}\\
		Supervisor: Fei Han, Department of Mathematics, National University of Singapore.
		
		\par This is a research module on the topic of Topological Data Analysis (TDA). The aim of the project is to investigate the effectiveness of applying methods in the emerging field of TDA to solve problems that involve natural language data. We will study the state-of-the-art method of persistent homology and several notable applications in computational analysis of natural language. In particular, we hope to investigate how effective the topological approach is compared to the traditional regression-based tools and why, which is still an open problem (Savle et al., 2019). In this project, we study the mathematical foundations of the state-of-the-art methods in topological data analysis as well as the existing semantic space models. Further, we aim to investigate the effectiveness of these methods as well as to propose and possibly implement some improvements. 
		
		\par Topics to be studied include: point-set topology, algebraic topology, persistence, homology, simplicial  complex, the persistence algorithms, semantic space models.
		
		\item (May 2020 -- Aug 2020)  \emph{\textbf{Numerical Exponential Polynomials.}}\\
		\hspace{0.2in}Supervisor: David Andrew Smith, Yale-NUS College.
		
		\par This project focused on the relative growth of terms in exponential polynomials. In different regions of the complex plane, different terms of an exponential polynomial dominate others. I used analytic and geometric arguments to compare the relative dominance of terms.
		
		For research outputs, please refer to my \href{https://github.com/zhangliu6/pdfs/blob/master/rank-terms-liu.pdf}{report} and  \href{https://github.com/zhangliu6/pdfs/blob/master/notes-on-complex-analysis-liu.pdf}{expository notes on Complex Analysis}.
		
		\item (Jul 2019 -- Sep 2019) \emph{\textbf{Hybrid Analytical-Numerical Integration.}}\\
		Supervisor: David Andrew Smith, Yale-NUS College.
		
		\par This project focused on Julia library for exponential sums and survey of Langer's work. I implemented a Julia library providing for the efficient description and algebra of exponential sums. Exponential sums are exponential polynomials in which each polynomial coefficient is a complex constant. I also surveyed Langer's (1931) paper on the asymptotic locus of zeros of such functions.
	 	
	 	For research outputs, please refer to my \href{https://www.unifiedtransformlab.com/outputs.html#Zha2019a}{report} and \href{https://www.unifiedtransformlab.com/outputs.html#Zha2019b}{code}.
	 	
		\item (Jun 2019 -- Jul 2019) \emph{\textbf{Digital Gazetteer Project at Yale Digital Humanities Lab.}}\\
		\hspace{0.2in}Supervisor: Catherine DeRose, Yale University.
		
		\par In this project worked with my colleagues at the Yale Digital Humanities Lab on the Digital Gazetteer Project. I built the geocoding database for the project and acquired skills in essential tools in Digital Humanities, including text mining and processing, ArcGIS, Augmented Reality using Unity game engine, user design principles. I also attended weekly meetings and various workshops organized by the Yale Digital Humanities Lab and the Yale Library. 
		
		\item (Sep 2018 -- Jun 2019)  \emph{\textbf{Ecological Modeling Research Cluster.}}\\
		\hspace{0.2in}Supervisor: Maurice Cheung, Yale-NUS College.
		 
		\par In this project, my colleagues and I employed mathematical modeling methods to analyze empirical ecological data using the Python programming language. Refer to \href{https://github.com/mauriceccy/ecologicalmodelling}{the git repository} for the project.
		
		\item (Jan 2018 -- Jun 2018)   \emph{\textbf{Emotion recognition from electroencephalography (EEG) signals using hybrid deep neural networks.}} \\
		 \hspace{0.2in} Supervisor: Hongliang Ren, School of Engineering, National University of Singapore.
		 
		 \par I worked full-time as a research intern at Lab of Medical Mechatronics, National University of Singapore (Department of Biomedical Engineering). I worked on an independent project on emotion recognition from electroencephalography (EEG) signals, using hybrid deep neural networks as well as transfer learning. Please refer to \href{https://sites.google.com/view/liu-zhang/home}{the project webpage} for more details.
		 
		
		\item (Apr 2016 --Dec 2017) \emph{\textbf{Application of Machine Learning in Automatic Sentiment Recognition in Human Speech}} \\
		 \hspace{0.2in}Supervisor: Yin Kwee Ng, Eddie, College of Engineering, Nanyang Technological University.
		 
		 \par  This project explored the possibility of applying supervised Machine Learning in recognizing sentiments in English utterances on a sentence level. In addition, the project also aimed to examine the effect of combining acoustic and linguistic features on classification accuracy. 
		 
		 \par With this project, I was among the International Finalists in the most prestigious science competition in China, the Dongrun-Yau Science Award organized by Tsinghua University. (Fewer than ten international finalists were selected per year.)
		 
		 \par The paper for this research project was published on Dec 27, 2018 as part of the Advances in Intelligent Systems and Computing book series (AISC, volume 887). Please refer to \href{https://link.springer.com/chapter/10.1007/978-3-030-03405-4_41}{the paper link} and \href{https://www.youtube.com/watch?v=fwrFHwDvMnU&list=PLS161HmXR6JjDQjE-lja23OKrbkJeO0vW&index=16}{the conference interview}for more details.
		 
		 \item (2015 -- 2017) \emph{\textbf{Training for the Singapore National Olympiad in Informatics and obtained Bronze medals.}}
		 
		 \par The training for the National Olympiad in Informatics helped me establish a solid foundation in competitive programming in C/C++ and developed my algorithm design and problem solving skills. 
		 
		 \item (Long-term Personal Passion Project) \emph{\textbf{The Panini Machine.}}
		 
		 \par Overview 
		 
		 In this project, I will attempt to use mathematical and computational methods to implement the Sanskrit linguistic system. The Sanskrit language was first formalized by the Indian linguist Panini (and hence the name of this project). My plan is to study the mechanics of Sanskrit and at the same time the relevant technical methods which I have broadly classified into the probabilistic approach and the deterministic approach. 
		 
		 I think of this project as a convergence of my interests. If we consider each of the disciplines a wood plank stacked on top of another, then the current project is my ongoing experiment to locate the ``center of gravity" where a balanced state can be achieved. 
		 
		 Significance 
		 
		 The significance of this project is two-fold. First, there is an increased interest in ancient Indian thought and traditions. However, many scholars are deterred by the lack of readily translated texts and the difficulties involved in learning Sanskrit. By implementing the grammatical system of Sanskrit, it is hoped that we can contribute to the task of Machine Translation in Sanskrit. The second and more important significance is the potential contribution to the current theories in Natural Language Processing (NLP), and more generally Artificial Intelligence. Most of the methods in NLP make use of probabilistic models to deal with the uncertainties in languages. Unlike most of these commonly spoken languages, Sanskrit is very peculiar in that it is highly systematic and even deterministic in nature, which could lead to some interesting alternatives to this probabilistic way of thinking. 
		 
		 For more details, see \href{https://zhangliu.gitbook.io/the-panini-machine/}{the gitbook documentation}.
		 
	\end{itemize}

\section{Computing Skills}

In this section, I list the programming languages and softwares I have had extensive experience with (i.e., having completed at least a substantial project).

\begin{itemize}
	\item \emph{\textbf{Julia:}}
	As a part of the summer research project in 2019, I studied the Julia programming language and implemented a Julia library. Specifically, this project focused on Julia library for exponential sums and survey of Langer's work. I implemented a Julia library providing for the efficient description and algebra of exponential sums. Exponential sums are exponential polynomials in which each polynomial coefficient is a complex constant. I also surveyed Langer's (1931) paper on the asymptotic locus of zeros of such functions. A sample of the code produced can be found \href{https://www.unifiedtransformlab.com/outputs.html#Zha2019b}{here}.
	
	\item \emph{\textbf{Python:}}
	I have completed the widely recognized DeepLearning.ai specialization on Coursera. (\href{https://www.coursera.org/account/accomplishments/specialization/3PAXFXZG7WAZ}{certificate link}) All the coding assignments in this specialization were written in Python. The course was taught by Professor Andrew Ng from Stanford University and consisted of the following courses: Sequence Models, Neural Networks and Deep Learning, Convolutional Neural Networks, Structuring Machine Learning Projects, Improving Deep Neural Networks: Hyperparameter Tuning, Regularization and Optimization.  Having completed this specialization, I was equipped with skills in implementing ML algorithms in Python Jupyter Notebook with Tensorflow and Keras back-end. 
	
	I also used Python extensively during my full-time research internship at NUS Lab of Mechatronics. I worked on an independent project on emotion recognition from electroencephalography (EEG) signals, using hybrid deep neural networks as well as transfer learning. Python is the langauge used for implementing the deep neural networks. Please refer to \href{https://sites.google.com/view/liu-zhang/home for more details}{the project website}. 
	
	Further, I was part of the Ecological Modeling Research Cluster where my colleagues and I employed mathematical modeling methods to analyze empirical ecological data using the Python. 
	
	Finally, I’m currently taking a course on Data Structures and Algorithms which use Python and C/C++ for assignments and exercises.  
	
	\item \emph{\textbf{C/C++:}}
	I have used C/C++ extensively in my participation in programming competitions in China and in Singapore since middle school. Notably, I have obtained Bronze medals in National Olympiad in Informatics in Singapore in 2015, 2016, and 2017. Further, I’m currently taking a course on Data Structures and Algorithms which use Python and C/C++ for assignments and exercises. 
	
	\item \emph{\textbf{R:}}
	I have used R for assignments and coding projects in the course Quantitative Reasoning, which is equivalent to an introductory course in Statistics. I have also used R extensively for a proof-based linear algebra course, for which I am currently the peer tutor. The course culminated into the final assignment where I implemented the QR algorithm and Single Value Decomposition of a general matrix. (\href{https://github.com/zhangliu6/course-submissions/blob/main/Coding1-5.zip}{code link})
	
	\item \emph{\textbf{OCaml:}}
	The Introduction to Computer Science class at our college was taught in the functional programming language OCaml (a statically type-checked programming language with an emphasis on expressiveness and safety). The course culminated in the final coding project written in OCaml on the following topics: multiplying integers in a tree using an accumulator; implementing depth-first and breadth-first traversals; indexing strings, arrays, lists, lazy lists, streams, and binary trees; computing the width of a binary tree; implementing a sieve; solving a variant of the knapsack problem; implementing an interpreter, a compiler, and a virtual machine for arithmetic expressions. 
	
	Refer to \href{https://github.com/zhangliu6/course-submissions/blob/main/Intro_to_CS_Midterm%20(7).pdf}{the midterm project report} and \href{https://github.com/zhangliu6/course-submissions/blob/main/midterm-codes.zip}{the midterm project code} as well as \href{https://github.com/zhangliu6/course-submissions/blob/main/Term%20Project.pdf}{the term project report} and \href{https://github.com/zhangliu6/course-submissions/blob/main/final-codes.zip}{the term project code}.
		
	\item \emph{\textbf{Coq Proof Assistant: }}
	I took the functional programming and proving class where the Coq Proof Assistant was used extensively. We studied an integrated account of specifications, unit tests, implementations, and properties of functional programs, through a variety of examples. 
	
	This course culminated in the final project to formalize an interpreter for arithmetic expressions, a compiler from arithmetic expressions to byte code, and an interpreter for byte code (i.e., a virtual machine), and to prove that for any given arithmetic expression, interpreting this arithmetic expression and compiling this arithmetic expression and then running the resulting byte-code program yield the same result, be it a natural number or an error message. 
	
	Refer to \href{https://github.com/zhangliu6/course-submissions/blob/main/FPP_Midterm%20(4).pdf}{the midterm project report} and \href{https://github.com/zhangliu6/course-submissions/blob/main/midterm-project-liu-new.v}{the midterm project code} as well as \href{https://github.com/zhangliu6/course-submissions/blob/main/FPP_Term_Project_Report.pdf}{the term project report} and \href{https://github.com/zhangliu6/course-submissions/blob/main/term-project-liu-1.v}{the term project code}.
		
	\item \emph{\textbf{MATLAB: }}
	I worked on a project on emotion recognition from electroencephalography (EEG) signals using hybrid deep neural networks as well as transfer learning. The benchmark dataset in the field is the DEAP dataset, a multimodal dataset for the analysis of human affective states. During the feature extraction stage, I used the EEGLAB package in MATLAB to analyze and post-process the EEG and peripheral physiological signals in batch. 
\end{itemize}
\bibliographystyle{amsplain}
{\small\bibliography{../dbrefs}}
\end{document}