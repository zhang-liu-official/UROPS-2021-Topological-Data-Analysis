% Preamble
\documentclass[11pt,reqno,oneside,a4paper]{article}
\usepackage[a4paper,includeheadfoot,left=25mm,right=25mm,top=00mm,bottom=20mm,headheight=20mm]{geometry}
%%%% DO NOT EDIT THIS FILE

% Standard packages
\usepackage{amssymb,amsmath,amsthm}
\usepackage{xcolor,graphicx}
\usepackage{verbatim}
\usepackage{mathtools}
\usepackage{hyperref}
% Layout of headers & footers
\usepackage{titling}
\usepackage{fancyhdr}
\newcommand{\runningtitle}{Running Title}
\pagestyle{fancy} \lhead{{\theauthor}} \chead{} \rhead{{\runningtitle}} \lfoot{} \cfoot{\thepage} \rfoot{}

% Hyphenation
\hyphenation{non-zero}

% Theorem definitions in the amsthm standard
\newtheorem{thm}{Theorem}
\newtheorem{lem}[thm]{Lemma}
\newtheorem{sublem}[thm]{Sublemma}
\newtheorem{prop}[thm]{Proposition}
\newtheorem{cor}[thm]{Corollary}
\newtheorem{conc}[thm]{Conclusion}
\newtheorem{conj}[thm]{Conjecture}
\theoremstyle{definition}
\newtheorem{defn}[thm]{Definition}
\newtheorem{cond}[thm]{Condition}
\newtheorem{asm}[thm]{Assumption}
\newtheorem{ntn}[thm]{Notation}
\newtheorem{prob}[thm]{Problem}
\theoremstyle{remark}
\newtheorem{rmk}[thm]{Remark}
\newtheorem{eg}[thm]{Example}
\newtheorem*{hint}{Hint}

%% Mathmode shortcuts
% Number sets
\newcommand{\NN}{\mathbb N}              % The set of naturals
\newcommand{\NNzero}{\NN_0}              % The set of naturals including zero
\newcommand{\NNone}{\NN}                 % The set of naturals excluding zero
\newcommand{\ZZ}{\mathbb Z}              % The set of integers
\newcommand{\QQ}{\mathbb Q}              % The set of rationals
\newcommand{\RR}{\mathbb R}              % The set of reals
\newcommand{\CC}{\mathbb C}              % The set of complex numbers
\newcommand{\KK}{\mathbb K}              % An arbitrary field
% Modern typesetting for the real and imaginary parts of a complex number
\renewcommand{\Re}{\operatorname*{Re}} \renewcommand{\Im}{\operatorname*{Im}}
% Upright d for derivatives
\newcommand{\D}{\ensuremath{\,\mathrm{d}}}
% Upright i for imaginary unit
\newcommand{\ri}{\ensuremath{\mathrm{i}}}
% Upright e for exponentials
\newcommand{\re}{\ensuremath{\mathrm{e}}}
% abbreviation for \lambda
\newcommand{\la}{\ensuremath{\lambda}}
% Make epsilons look more different from the element symbol
\renewcommand{\epsilon}{\varepsilon}
% Always use slanted forms of \leq, \geq
\renewcommand{\geq}{\geqslant}
\renewcommand{\leq}{\leqslant}
% Shorthand for "if and only if" symbol
\newcommand{\Iff}{\ensuremath{\Leftrightarrow}}
% Make bold symbols for vectors
\providecommand{\BVec}[1]{\mathbf{#1}}
% Hyperbolic functions
\providecommand{\sech}{\operatorname{sech}}
\providecommand{\csch}{\operatorname{csch}}
\providecommand{\ctnh}{\operatorname{ctnh}}
% sinc function
\providecommand{\sinc}{\operatorname{sinc}}
% closure of a set
\providecommand{\clos}{\operatorname{clos}}
% The absolute value of a real number or modulus of a complex number, with automatically scaling delimiters
\newcommand{\abs}[1]{\left\lvert#1\right\rvert}
\newcommand{\sgn}{\operatorname{sgn}}

% add two sub and superscripts with a space between them
\newcommand{\Mspacer}{\;} %Spacer for below Matrix display functions
\newcommand{\M}[3]{#1_{#2\Mspacer#3}} %Print a symbol with two subscripts eg a matrix entry
\newcommand{\Msup}[4]{#1_{#2\Mspacer#3}^{#4}} %Print a symbol with two subscripts and a superscript eg a matrix entry
\newcommand{\Msups}[5]{#1_{#2\Mspacer#3}^{#4\Mspacer#5}} %Print a symbol with two subscripts and two superscripts eg a matrix entry
\newcommand{\MAll}[7]{\prescript{#1}{#2}{#3}_{#4\Mspacer#5}^{#6\Mspacer#7}} %Print a symbol with two subscripts and two superscripts eg a matrix entry

% Make really wide hat for Fourier transforms applied to large functions
\usepackage{scalerel}
\usepackage{stackengine}
\stackMath
\newcommand\reallywidecheck[1]{%
\savestack{\tmpbox}{\stretchto{%
  \scaleto{%
    \scalerel*[\widthof{\ensuremath{#1}}]{\kern-.6pt\bigwedge\kern-.6pt}%
    {\rule[-\textheight/2]{1ex}{\textheight}}%WIDTH-LIMITED BIG WEDGE
  }{\textheight}%
}{0.5ex}}%
\stackon[1pt]{#1}{\scalebox{-1}{\tmpbox}}%
}
\providecommand{\widecheck}{\reallywidecheck}

\newcommand\reallywidehat[1]{%
\savestack{\tmpbox}{\stretchto{%
  \scaleto{%
    \scalerel*[\widthof{\ensuremath{#1}}]{\kern-.6pt\bigwedge\kern-.6pt}%
    {\rule[-\textheight/2]{1ex}{\textheight}}%WIDTH-LIMITED BIG WEDGE
  }{\textheight}%
}{0.5ex}}%
\stackon[1pt]{#1}{\tmpbox}%
}


%% Acknowledgements
\newcommand{\AckYNCSRP}[1]{#1 gratefully acknowledges support from Yale-NUS College summer research programme.}
\newcommand{\AckYNCProj}[1]{#1 gratefully acknowledges support from Yale-NUS College project B grant IG18-PRB102.}
\newcommand{\AckYNCWorkshop}[1]{#1 gratefully acknowledges support from Yale-NUS College workshop grant IG18-CW003.}
\newcommand{\AckNICA}[1]{#1 would like to thank the Isaac Newton Institute for Mathematical Sciences for support and hospitality during programme \emph{Complex analysis: techniques, applications and computations}, when work on this paper was undertaken. This work was supported by EPSRC Grant Number EP/R014604/1.}
\newcommand{\AckSMRIIVP}[1]{#1 would like to thank the Sydney Mathematics Research Institute for support and hospitality under the International Visitor Programme.}
 % Standard packages, page layout, theorem environments, macros, etc
% This file contains macros specific to the project.
% You are welcome to add your own macros, but please avoid deleting those written by others.

% Asymptotic notation
\newcommand{\bigoh}{\mathcal{O}}
\newcommand{\lindecayla}{\bigoh\left(\abs{\la}^{-1}\right)}
 % Macros specific to this project.
\usepackage{bbm}
\DeclareMathOperator{\Cech}{\check{C}}
\author{Zhang Liu}
\title{Notes on Topological Data Analysis}
\renewcommand{\runningtitle}{Notes on Topological Data Analysis}
\date{\today}

\begin{document}
\maketitle
\thispagestyle{fancy}

\begin{abstract}
    This document is to serve as a set of notes to fill the gaps in my understanding of Topological Data Analysis relevant to the project. The reference book for this set of notes is \textit{Algebraic Topology} \cite{Hat2002a}, \textit{Topology and Data} \cite{Car2009a}, and \textit{Topology for Computing} \cite{Zom2005a}.
\end{abstract}

% \tableofcontents

\section{Motivation}

First of all, there are four major advantages for using topological methods to deal with point clouds in data analysis.  

\begin{enumerate}
	\item Topology provides qualitative information which is required for data analysis.
	
	\item Metrics are not theoretically justified. Compared to straightforward geometric methods, Topology is less sensitive to the actual choice of metrics. 
	
	\item Studying geometric objects using Topology does not depend on the coordinates. 
	
	\item Functoriality. 
	
	\begin{defn}
		For any topological space X, abelian group $A$, and integer $k\geq 0,$ there is assigned a group $H_k(X,A).$
		
		For any $A$ and $k$, and any continuous map $f: X \to Y,$ there is an induced homomorphism $H_k(f,A): H_k(X,A) \to H_k(Y,A).$ Then \textit{functoriality} refers to the following conditions:
		
		\begin{itemize}
			\item  $H_k(f\circ g,A): H_k(f,A) \circ H_k(g,A)$
			\item  $H_k(Id_{X};A) = Id_{H_k(X,A)}.$
		\end{itemize}
	\end{defn}

	Functoriality addresses the ambiguities in statistical clustering methods - in particular the arbitrariness of various threshhold choices. We now illustrate how exactly functoriality could be used in questions related to clustering. 
	
	Let $X$ be the full data set and $X_1,X_2$ are the subsamples from the data set. If the set of clusterings $C(X_1), C(X_2), C(X_1 \cup X_2)$ correspond well (this notion will be defined formally in later section), then we can conclude that the subsample clusterings correspond to clusterings in the full data set $X$. 
\end{enumerate}

\section{Homotopy} \label{sec:homotopy}

\begin{defn}
	\textit{Homotopy} is a family of maps $f_t: X \to Y$ where $t \in I$ such that $F: X\times I \to Y$ defined by $F(x,t) \mapsto f_t(x)$ is continuous.
\end{defn}

\begin{defn}
	Two maps $f_0, f_1$ are \textit{homotopic} if $\exists$ a homotopy $f_t$ between $f_0$ and $f_1$.  
\end{defn}

A special case of homotopy is the deformation retraction. 

\begin{defn}
	A \textit{deformation retraction} of $X$ onto a subspcae $A$ is a homotopy from the identity map of $X$ to a retraction of $X$ onto $A$, $r: X\to X$ such that $r(X) = A$ and $r|_A = \mathbbm{1}$ (or equivalently, retraction is the map $r: X\to X, r^2 =r$). 
\end{defn}

Retraction is the topological analog of projection. To visualize this analogy, we give an example of how some deformation retractions arise from the mapping cylinder.

\begin{defn}
	For a map $f: X\to Y$, the \textit{mapping cylinder} $M_f$ is the quotient space of the disjoint union $(X \times I) \sqcup Y$.
\end{defn}

\begin{defn}
A map $f:X \to Y$ is a homotopy equivalence if there is a map $g: Y \to X$ such that 
\begin{itemize}
	\item $f\circ g$ is homotopic to the identity map on $Y$, and
	\item $g\circ f$ is homotopic to $f$.
\end{itemize}

Two spaces $X,Y$ are\textit{ homotopy equivalent} if there exiss a homotopy equivalence $f: X \to Y. $
\end{defn}

\begin{defn}
	If $f$ and $g$ are homotopic, then $H_k(f,A) = H_k(g,A).$ Then it follows that if $X$ and $Y$ are homotopy equivalent, then $H_k(X,A) \cong H_k(Y,A) $.
\end{defn}

\begin{defn}
	For any field $F$, $H_k(X,F)$ will be a vector space over $F$. Then if $F$ is finite dimensional, its dimension is referred to as the $k$-th Betti number with coefficients in $F$, denoted as $\beta_{k}(X,F).$
	
	The $k$-th Betti number corresponds to an informal notion of the number of independent $k$-dimensional surfaces. If two spaces are homotopy equivalent, then all their Betti numbers are equal. 
	
	Note that the Betti numbers can vary with the choice of the coefficients in $F$. 
\end{defn}


\begin{defn}
	An \textit{abstract simplicial complex} is a pair $(V, \Sigma)$, where $V$ is a finite set, and $\Sigma$ is a family of non-empty subsets of $V$ such that 
	$$\sigma \in \Sigma, \tau \subseteq \sigma \implies \tau \in \Sigma.$$
	
	Associated to a simplicial complex is a topological space $|(V,\Sigma)|$. $|(V,\Sigma)|$ may be defined using a bijection $\phi : V \to \{1, 2, \dots,N\}$ as the subspace of $\mathbb{R}^N$ given by the union 
	$$\cup_{\sigma \in \Sigma} c(\sigma),$$
	where $c(\sigma)$ is the convex hull of the set $\{e_{\phi(s)}\}_{s\in \sigma}$, where $e_i$ denotes the $i$th standard basis vector.
\end{defn}

	 We often use abstract simplicial complexes to approximate topological spaces. For simplicial complexes the homology can be computed using only the linear algebra of finitely generated $\mathbb{Z}$-modules. In particular, for simplicial complexes, homology is algorithmically computable (unlike the standard methods for computing the Smith normal form).

\begin{defn}
	Let $X$ be a topological space, and let $\mathcal{U} = \{U_\alpha\}_{\alpha\in A}$ be any covering of $X$. 
	
	The \textit{nerve} of $\mathcal{U}$, denoted by $N\mathcal{U}$, will be the abstract simplicial complex with vertex set $A$, and where a family $\{\alpha_0, \dots, \alpha_k\}$ spans a $k$-simplex if and only if $U_{\alpha_0}\cap U_{\alpha_1} \cap \cdots \cap U_{\alpha_k} \neq \emptyset.$
\end{defn}

One reason that this construction is very useful is the following ``Nerve Theorem." This theorem gives the criteria for $N(\mathcal{U})$ to be homotopy equivalent to the underlying topological space $X$.

\begin{thm}
Suppose that $X$ and $U$ are as above, and suppose that the covering consists of open sets and is numerable. Suppose further that for all $\emptyset \subseteq A,$ we have that $\bigcap_{s\in S} U_s$ is either contractible or empty. Then $N(\mathcal{U})$ is homotopy equivalent to $X$.
\end{thm}

\begin{defn}
	For any subset $V\subseteq X$ for which $X = \bigcup_{v\in V}B_\epsilon(v)$, one can construct the nerve of the covering $\{B_\epsilon(v)\}_{v\in V}$. This construction is referred to as the ``\v{C}ech complex" attached to V and is denoted as $\Cech(V,\epsilon)$.
\end{defn}

\begin{thm}
Let $M$ be a compact Riemannian manifold. Then there is a positive number $e$ so that $\Cech(M, \epsilon)$ is homotopy equivalent to $M$ whenever $\epsilon \leq e$. Moreover, for every $ \epsilon \leq e$, there is a finite subset $V \subseteq M$ so that the subcomplex of $\Cech(V,\epsilon) \subseteq \Cech(M, \epsilon)$ is also homotopy equivalent to $M$.
\end{thm}

However, this construction is computationally expensive. A solution is to construct a simplicial complex which can be recovered solely from the edge information, which motivates the following construction known as the ``Vietoris-Rips complex." 

\begin{defn}
	Let $X$ be a metric space with metric $d$. Then the  \textit{Vietoris-Rips complex} for $X$, attached to the parameter $\epsilon$, denoted by $VR(X,\epsilon)$, will be the simplicial complex whose vertex set is $X$, and where $\{x_0,\dots, x_k\}$ spans a $k$-simplex if and only if $d(x_i,x_j) \leq \epsilon$ for all $0\leq i,j\leq k.$
\end{defn}

\begin{prop}
	Comparing the \v{C}ech complex and the VR compelx:
	$$ \Cech(X,\epsilon) \subseteq VR (X,2\epsilon) \subseteq \Cech(X,2\epsilon).$$
\end{prop}

However, even the VR complex is computationally expensive. A solution, again, is to the Voronoi decomposition which studies the subspaces of Euclidean space. 

\begin{thm}
	Let $X$ be any metric space, and let $\mathcal{L}\subseteq X$ be a subset (called the set of \textit{landmark points}). Given $\lambda \in \mathcal{L}$, we define the \textit{Voronoi cell} associated to $\lambda, V_\lambda$, by
	$$V_\lambda = \{x\in X| d(x,\lambda) \leq d(x,\lambda^\prime)\} \forall \lambda^\prime \in \mathcal{L}.$$
\end{thm}


\begin{defn}
	Similar to how we define the \v{C}ech complex above, we define the Delaunay complex attached to $\mathcal{L}$ to be the nerve of this covering.
\end{defn}

However, for finite metric spaces, the Delaunay complex generically produces degenerate (i.e. discrete) complexes with no $1$-simplices. To solve this, we modify the definition to accommodate pairs of points which are ``almost" equidistant from a pair of landmark points. We thus have the definition below: 
\begin{defn}
	Let X be any metric space, and suppose we are given a finite set $\mathcal{L}$ of points in $X$ (called the landmark set), and a parameter $\epsilon> 0$. For every point $x \in X$, we let $m_x$ denote the distance from this point to the set $\mathcal{L}$, i.e., the minimum distance from $x$ to any point in the landmark set.
	
	Then we define the\textit{strong witness complex} attached to this data to be the complex $W^s(X,\mathcal{L}, \epsilon)$ whose vertex set is $\mathcal{L}$, and where a collection $\{l_0, \dots, l_k\}$ spans a $k$-simplex if and only if there is a point $x \in X$ (the witness) so that $d(x, l_i)\leq m_x + \epsilon$ for all $i$. 
	
	We can also consider the version of this complex in which the $1$-simplices are identical to those of $W(X,\mathcal{L},\epsilon)$, but where the family $\{l_0, \dots, l_k\}$ spans a $k$-simplex if and only if all the pairs $(l_i, l_j)$ are $1$-simplices. We will denote this by $W^s_VR$.
\end{defn}

A modified version of the strong witness complex is also useful:

\begin{defn}
	We construct the\textit{weak witness complex}, $W^w(X,\mathcal{L}, \epsilon)$, attached to the given data by declaring that a family $\Lambda = \{l_0, \dots, l_k\}$ spans a $k$-simplex if and only if $\Lambda$ and all its faces admit $\epsilon$ weak witnesses. 
	
	Similar to the definition for strong witness complex, we can also consider the version of the weak witness complex in which the $1$-simplices are identical to those of $W(X,\mathcal{L},\epsilon)$, but where the family $\{l_0, \dots, l_k\}$ spans a $k$-simplex if and only if all the pairs $(l_i, l_j)$ are $1$-simplices. We will denote this by $W^w_VR$.
\end{defn}

\bibliographystyle{amsplain}
{\small\bibliography{../dbrefs}}
\end{document}
