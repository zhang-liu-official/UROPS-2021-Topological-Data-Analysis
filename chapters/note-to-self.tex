To compute the homological properties of a point cloud,
\cite{ELZ_02} introduced persistent homology, refined by
\cite{Carlsson_04}. Using one of a whole family of methods, the point
cloud induces a filtered simplicial complex, where the filtration
encodes distance data as increasing ``closeness'' data for the data
points in the point cloud.


What's the diff between with tilde on top vs without?

cycle group should be closed $C_k^i$ vs chain group $\Chain_k^i$??

simplicial complex vs covering vs nerve theorem 

where does nerve theorem come in?

how do we make a narrative? can we just say explorative study?

how to interpret the barcode and persistence diagrams??
features for a point cloud data


advantages: 
- independent of coordinates on the neural manifolds
- probably will be independent of the dimensionality reducion/filtering function used???


difficulties:
- diffusion map: setting parameters is tricky!!


distance between metric spaces
spaces of fingerprints a fingerprint is a metric space 

wasserstein

why is wasserstein better??

experiment 
simulations
This chapter I follow this book. 
Wasserstein: what kind of distance we have?

this distance -> this literature 
that distance -> that literature
wass -> follow geometry paper

comments and intuitions heuristic comments

remarks
limitations & advantages